\documentclass[12pt]{scrartcl}
\usepackage[margin=2cm
  %,showframe% <- only to show the page layout
]{geometry}
\renewcommand*{\titlepagestyle}{empty}

\title{Detecting subtle text manipulations: a cross-article analysis chasing the signals of media framing}
\author{Martino Mensio}
\date{1 April 2020}

\begin{document}

\maketitle

\begin{abstract}
% Orientation
In the world of public misinformation, there are many cases where the information is not false or fabricated, but rather has been manipulated using subtle techniques such as word replacements, omissions and argument distortion. These techniques can have a considerable effect in influencing the reader’s frame of mind towards the events reported.
% Rationale
Studies have analysed the problem of relating news articles that talk about the same events, in order to find information that appears in multiple articles or has been omitted. However, there is a lack of research that enrich this comparative analysis with signals of media framing.
% Aim
This research aims to propose and develop a framework that allows to automatically put in contrast the differences in the framing of articles that talk about the same events.
% Method
The method relies on Natural Language Processing techniques to find similar articles and sentences and then extract framing indicators, such as specific word choices, subjectivity measures and argument analysis. The extracted properties are then attached to a similarity graph that represents the relatedness of articles and subparts of them. In this way, we explore and compare different articles that present the same information in different ways, highlighting for example details that have been selected or omitted or differences in the language strength and subjectivity.
% Findings
The initial results show that many news outlets are reusing and modifying content from others with small changes. We need to investigate further the nature of this phenomenon and understand the motivations behind it.
% Interpretation
This work can bring a contribution in analysing how the information is copied, modified or simply presented differently. The comparison of media framing of different articles could enable readers to have a wider perspective on the events reported and help them to reveal the choices that have been done while presenting the information.
\end{abstract}
%----

\end{document}